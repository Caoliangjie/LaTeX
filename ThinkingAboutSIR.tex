%!Mode::"Tex:UTF-8"
\documentclass{article}
\bibliographystyle{plain}
\usepackage{indentfirst}
\usepackage{picinpar,graphicx}
\usepackage{cite}
\setlength{\parindent}{2em}
\author{Hongzhi Liu}
\title{Thinking About SIR}
\begin{document}
\maketitle
  \par
  These days, I spend much time reading PhD Wan's paper named Benchmarking Single-Image Reflection Removal Algorithms which aims at separating the reflection from the mixture image using one or more shots. It can be roughly divided into two parts. One is benchmark dataset and the other one is algorithms evaluation.

  The research starts from a brief survey of existing reflection removal algorithms. The single-image approach attracts great attention due to its simplicity in setup and practicability for non-professional users. So the SIR2 benchmark dataset has been proposed in the paper that contains more than a thousand images.

   Besides, they use the SIR2 dataset to evaluate representative single-image reflection removal algorithms for both quantitative accuracy and visual quality. The quantitative evaluation is performed by checking the difference between the ground truth of background L$_B$ and the estimated background L*$_B$ from each four method that are AY07, LB14, SK15 and WS16.

   Furthermore, they realize that the single-image reflection removal algorithm still has great space to be improved and many reflections only occupy a part of the whole images in the dataset. Methods that automatically detect and process the reflection regions may have potential to improve the overall quality \cite{Wan2017Benchmarking}.

\bibliographystyle{IEEEtran}
\bibliography{1}

\end{document} 