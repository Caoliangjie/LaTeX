%!Mode::"Tex:UTF-8"
\documentclass{article}
\usepackage{indentfirst}
\setlength{\parindent}{2em}
\author{Henry Liu}
\title{Learn about single-pixel multispectral imaging}
\begin{document}
  \maketitle
  \section{What is single-pixel imaging}
  \par 
  Combining spectral imaging with compressive sensing (CS) enables efficient data acquisition by fully utilizing the intrinsic redundancies in natural images. Current compressive multispectral imagers, which are mostly based on array sensors (e.g, CCD or CMOS), suffer from limited spectral range and relatively low photon efficiency. 
  
  To address these issues, there is a multispectral imaging scheme with a single-pixel detector. Inspired by the spatial resolution redundancy of current spatial light modulators (SLMs) relative to the target reconstruction, people design an all-optical spectral splitting device to spatially split the light emitted from the object into several counterparts with different spectrums. Separated spectral channels are spatially modulated simultaneously with individual codes by an SLM. 
  
  This no-moving-part modulation ensures a stable and fast system, and the spatial multiplexing ensures an efficient acquisition. A proof-of-concept setup is built and validated for 8-channel multispectral imaging within 420~720 nm wavelength range on both macro and micro objects, showing a potential for efficient multispectral imager in macroscopic and biomedical applications.
  \section{What we can learn from it}
  \par
  From the thesis of Jinli Suo, we can know a new technology that can be fully used in the field of medical.We can apply this tech to the research of material identification and biological observation.
  
  Furthermore,better than what we learn called Nyquist-sampling Theorem, Compressive sensing is an more efficient signal acquisition scheme,which can be applied to reduce the acquisition time.

\end{document}